\title{Introductions to Objects}
\documentclass[12pt]{article}
\begin{document}

\title{FIT1002 : Modularization}
\date{27 August, 2012 - 28 August,2012 }
\author{Gordon Ng}
\maketitle

\pagebreak
\tableofcontents
\pagebreak

\section{UML}
\begin{itemize}
\item Upper Box: Name of Class
\item Middle box:Attributes
\item Lower box:
\item + Public Class
\item - Private Class
\item method name, type
\end{itemize}
\subsection{Example}
+ test(): void

\section{Encapsulation}
It allows data and code that manipulates data to be visible or invisible from code outside the class.

\section{Visibility}
A instance variable or a method can be private which means only code inside the class. 

There should be only one main method for each project.

\section{Our own classes}
There should be no main method and no static after public. \\
Static = I can do something.
\section{Instance Variable}
Their scope is the entire class they are declared in. Their lifetime is until they are destroyed by the garbage collector or any code.


\section{Reminder}
Rectangle r = create a reference variable that can point at any Rectangle2 object. It cannot point at anything other than a rectangle. \\
r = new rectangle(2,5) is \\ 
Initally it points at NULL. (This causes a NULL pointer error.  )

Two reference variables that points at the same object.
\end{document}