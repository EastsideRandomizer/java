\documentclass[12pt]{article}
\begin{document}
\title{FIT1002 : Methods Revisited}
\date{27 August, 2012 - 28 August,2012 }
\author{Gordon Ng}
\maketitle
\pagebreak
\tableofcontents
\pagebreak

\section{Parameters}
\begin{itemize}
\item It allows us to pass data into methods.
\item It makes the methods more flexible.
\item Example:\\
 private static char calc(int num1,int num2, String message).
\item The variables declared inside the definition of method are known as FORMAL PARAMETER LIST .
\item In a FORMAL PARAMETER LIST, it cannot contain an expression.
\item The data that resides inside a method call is known as ACTUAL PARAMETER. It also must be a compatible type towards the FORMAL PARAMETER.
\end{itemize}
\subsection{Lifetime}
Their lifetime is the same as a local variable.

\subsection{Constructors}
The default constructor is parameter-less.

\pagebreak
\subsection{Overloading}
One can have multiple methods with the \textbf{same} name as long as they have different signatures. \\
The signature contains the method name AND the FORMAL parameter list.
For example:
\begin{itemize}
\item test(int a, int b);
\item test(int c);
\end{itemize}
Both methods are legal as they have different Formal parameters. \\
That way we can do Constructor overloading to create new objects:
\begin{itemize}
\item public Car(int Cost);
\item public Car(int Age, int Cost);
\end{itemize}
Both can be used to create different objects.

\section{Static Method}
It doesn't understand objects. Does it matter?
\end{document}

