\documentclass[12pt]{article}
\begin{document}

\section{Lecture 3 - Algorithms and Data Type}
\begin{itemize}
\item A list if well defined insturction for completing a task.
\item An expression of logic that can be implemented by a program.
\item A program is used to express the logic.
\item Examples: \textbf{A cooking recipe}.
\item They must be clear and well-defined.
\item They must stop after a finite steps.
\item Instructions \textbf{must} be performed in the \textbf{order they are given.}
\end{itemize}
\subsection{Example}
A cooking receipe

\subsection{importance}
\begin{itemize}
\item The algorithm should not be made at the expense of clarity (Can it be understood?)
\item It needs to be flexible
\item The Algorithm should be well documented.
\end{itemize}

\subsection{Pseudocode}
\begin{itemize}
\item It has a very simple and loose syntax.
\item It is chosen by the coder so that the syntax is not an issue.
\end{itemize}

\pagebreak 
\section{Data - Values}
When a data value appears in program code it is called literal
\begin{itemize}
\item + = concatrate
\item Anything inside a quotation mark is string
\end{itemize}
\subsection{Variables}
It is a named address in the computer's memory.
\begin{itemize}
\item Named containers for values.
\item Contains only \textbf{one} value at anytime.
\item Can only value of a \textbf{specified type}.
\end{itemize}
\subsubsection{Assignment Opreator}
\subsubsection{identifier}
One cannot use reserved keywords as a identifer.
\subsection{Constants}
\begin{itemize}
\item A constant similar to a variable except it must have an inital value
\item That value cannot be overwritten
\item final is a modifier
\end{itemize}
\subsubsection{Importance}
\begin{itemize}
\item Constants facilitaes program maintainence 
\end{itemize}

\pagebreak

\subsection{Primitive Data Types}
There are 8 data types 
\subsubsection{integers}
\begin{itemize}
\item byte
\item short 
\item int
\item long
\end{itemize}
\subsubsection{char}
Char is a datatype of single character.
They are enclosed in single quotes so they are not mistaken for variables.
(\textbf{N.B.:} String are enclosed in double quote)

\end{document}
