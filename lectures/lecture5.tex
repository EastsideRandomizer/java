\title{Introductions to Objects}
\documentclass[12pt]{article}
\begin{document}

\title{FIT1002 : Introduction to Objects}
\date{6 August, 2012}
\author{Gordon Ng}
\maketitle

\pagebreak
\tableofcontents
\pagebreak
\section{Objects}
Objects can have a state and can have behaviours
\subsection{Attributes}
An objects's data is often called its Attributes or Characteristics. \\
They are Implements in Java as Instance Variables.
e.g:Current bank balance
\subsection{Behaviours}
They are implemented in Java as methods

\section{Class}
A class is arecipe for making objects from the same Class.\\
We can create multiple Object from the same Class.
\begin{itemize}
\item An Object is defined by a Class.
\item All object of a Class have the same data item but different values.
\item All objects os a class has same behaviours.
\item An object is an instance of a Class.
\end{itemize}


\pagebreak
\section{Creating Reference Varibles}
Assuming there is a Class baned Student. We will declare a variable capable of referencing a student object.  \\
\\
Student aStudent;
\\ \\
We have created a variable that can contain the address of a Student's data and point at a student object.
\\
aStudent = new Student(``Mary Jane'',''mjane@student.com'',''Comp.Sci'')
\\
\begin{itemize}
\item aStudent: Reference Variable
\item new: instantiation opreator (Creates a student object)
\item Student: same name as the object's Class which is the same name as the object's reference type.
\item ():Constructor
\end{itemize}

\section{Java Standard Class Library}
\subsection{Example Classes}
\subsubsection{Scanner}
Scanner is a reference type variable (or a class type). \\
Scanner cannot be resolved as a primitive type.\\
Hence \textbf{java.util.Scanner} or \textbf{java.util.Scanner} or \textbf{java.util.*} is used. \\
A package lang is automatically imported to any java file.

\end{document}