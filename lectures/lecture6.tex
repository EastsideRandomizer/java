
\documentclass[12pt]{article}
\begin{document}

\title{FIT1002 : Introduction to Objects - Continued}
\date{7 August, 2012}
\author{Gordon Ng}
\maketitle

\pagebreak
\tableofcontents
\pagebreak
\section{Reference Variables}
int count = 38; \\
int = Primitive type. \\
38 is a value in the memory. \\
Reference \\
\subsection{Assignment}
It is same procedure as for primitive type except the value copied is an ADDRESS. \\
JVM undergoes Garbage Collection. 

\section{String Class}
String is a reference type but it seems like a primitive type (It can has any size and has to be stored in the heap).
\subsection{Common Methods}
\begin{itemize}
\item length():
\item CharAt(n):
\item toUpperCase:
\item toLowerCase:
\item substring($n_{lower},n_{upper}$)
\end{itemize}
\pagebreak
\subsection{Character Indexs and Concatentation}
\subsubsection{String Concatenation}
\pagebreak
\section{Math Class}
\begin{itemize}
\item It is called a static class. 
\item It is not an object.
\item They are relatively rare.
\item They contain the method of generating random number.
\item It also provide mathematical constant
\item \textbf{It is involked by Math.*}
\end{itemize}
\subsection{Common}
\begin{itemize}
\item sqrt (double num)
\end{itemize}

\section{Others}
\subsection{Decimal Format}
DecimalFormat places3 = new DecimalFormat(``0.\#\#''); \\
System.out,println(s)
\subsection{Enumerated Type}
It defines all possible value of a variable of a enumerated type.\\
\textbf{How to use:} \\
enum Grade (A,B,C,D,E,F); \\
\textbf{Assigning value:} \\
vairable = variablename.value; \\
\pagebreak
\section{Why we don't use NextLine}
End of Line Problems. Sometimes NextLine is used twice but it more advisible to not to use it.

\end{document}